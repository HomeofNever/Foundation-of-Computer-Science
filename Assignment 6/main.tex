\documentclass{article}
\usepackage[utf8]{inputenc}
\usepackage{datetime}
\usepackage{enumerate}
\usepackage{textcomp}
\usepackage{amsmath}
\usepackage{amssymb}
\usepackage{tikz}
\usetikzlibrary{arrows,shapes,backgrounds}
\usepackage[edges]{forest}
\usepackage{tikz}
\usetikzlibrary{arrows}

\usepackage{titlesec}
\newcommand{\sectionbreak}{\clearpage}

\title{\bf \Large ASSIGNMENT 6}
\author{Xinhao Luo} 
\date{\today}

\def\math#1{$#1$}

\setlength{\textheight}{8.5in}
\setlength{\textwidth}{6.5in}
\setlength{\oddsidemargin}{0in}
\setlength{\evensidemargin}{0in}
\voffset0.0in

\begin{document}
\maketitle
\medskip

\section{Problem 13.8}
\begin{enumerate}[a)]
    \item \math{26^5}
    \item \math{\frac{26!}{21!}}
    \item \math{26^2}
    \item \math{26^2 * 2}
    \item \math{26^2 * 2 - 1}
\end{enumerate}

\section{Problem 13.44}
\begin{enumerate}[a)]
    \item \math{\frac{10!}{4!6!} = 210}
    \item \math{\frac{10!}{(10 - 4)!} = 5040}
    \item \math{\binom{14 - 1}{10 - 1} = \frac{13!}{9!4!} = 715}
    \item \math{10^4}
\end{enumerate}

\section{Problem 13.50}
\begin{enumerate}[a)]
    \item \math{\binom{10 - 1}{4 - 1} = 84}
    \item \math{\binom{10 + 4 - 1}{4 - 1} = 286}
    \item \math{\binom{12 + 4 - 1}{4 - 1} = 455}
\end{enumerate}

\section{Problem 13.51(a)}
\begin{enumerate}
    \item The number of all possible combination of 20-bit binary string is \math{2^{20}}
    \item Set \math{Q(n)} as the number of 20-bit binary string does not contain "00"
    \item Base case: 
        \begin{enumerate}
            \item \math{Q(0) = 0}
            \item \math{Q(1) = 2}
            \item \math{Q(2) = 3}
            \item \math{Q(3) = 5}
        \end{enumerate}
    \item For \math{n \geq 4}, we have two cases:
        \begin{enumerate}
            \item String start from "1" \\
                So we have "1"+\math{Q(n - 1)}
            \item String start from "0" \\
                The second digit must be "1" \\
                So we have "01"+\math{Q(n - 2)}
        \end{enumerate}
    \item In conclusion, \begin{equation}
            \begin{split}
                Q(n) = Q(n - 1) + Q(n - 2)
            \end{split}
        \end{equation}
    \item The result will then be \math{2^{20} - Q(20)}, \math{Q(20) = 17711}
    \item 
\end{enumerate}

\section{Problem 13.61}

\math{a = 2x}, \math{b = \sqrt{x}}, \math{n = 10} \\

\begin{equation}
    \begin{split}
        \binom{10}{k} (2x)^k (\sqrt{x})^{10-k} &= \binom{10}{k} 2^kx^k x^{5+\frac{k}{2}} \\
        &= \binom{10}{k} 2^k x^{5+\frac{k}{2}}
    \end{split}
\end{equation}

\begin{itemize}
    \item [\math{x^3}] \math{5+\frac{k}{2} = 3}, \math{k = -4}, coefficient is \math{0} (section not exist)
    \item [\math{x^4}] \math{5+\frac{k}{2} = 4}, \math{k = -2}, coefficient is \math{0} (section not exist)
    \item [\math{x^5}] \math{5+\frac{k}{2} = 5}, \math{k = 0}, \math{\binom{10}{0} 2^0 x^5 = x^5}, the coefficient is 1
    \item [\math{x^6}] \math{5+\frac{k}{2} = 6}, \math{k = 2}, \math{\binom{10}{2} 2^2 x^6 = 180 x^6}, the coefficient is 180
    \item [\math{x^7}] \math{5+\frac{k}{2} = 7}, \math{k = 4}, \math{\binom{10}{4} 2^4 x^7 = 3360 x^5}, the coefficient is 3360
\end{itemize}

\section{Problem 14.5}
\begin{enumerate}[a)]
    \item \math{\binom{10 + 4 - 1}{4 - 1} = \frac{13!}{10!3!} = 286}
    \item \math{\frac{15!}{5!5!5!}}
    \item \math{\binom{10 + 10 - 1}{10 - 1}}
    \item \math{\frac{9!}{3!3!3!}}
\end{enumerate}

\section{Problem 14.14}
\begin{enumerate}[a)]
    \item \math{\binom{10}{6} + \binom{10}{7} + \binom{10}{8} + \binom{10}{9} + \binom{10}{10}}
    \item \math{2^5 + 2^4 * 5}
    \item \math{2^5 + 2^4 * 5}
    \item \math{(2^5 + 2^4 * 5) * 2 - 2}
\end{enumerate}

\end{document}