\documentclass{article}
\usepackage[utf8]{inputenc}
\usepackage{datetime}
\usepackage{enumerate}
\usepackage{textcomp}
\usepackage{amsmath}
\usepackage{amssymb}
\usepackage{tikz}
\usetikzlibrary{shapes,backgrounds}

\usepackage{titlesec}
\newcommand{\sectionbreak}{\clearpage}

\title{\bf \Large ASSIGNMENT 2}
\author{Xinhao Luo} 
\date{\today}

\def\math#1{$#1$}

\setlength{\textheight}{8.5in}
\setlength{\textwidth}{6.5in}
\setlength{\oddsidemargin}{0in}
\setlength{\evensidemargin}{0in}
\voffset0.0in

\begin{document}
\maketitle
\medskip

\section{Problem 3.53}
\begin{enumerate}[(a)]
    \item \math{x \in \mathbb{N}, \mathbb{Z}, \mathbb{Q}, \mathbb{R}}
    \item \math{x \in \mathbb{R}}
    \item 
        \begin{itemize}
            \item \math{x \in \mathbb{N}, y \in \mathbb{N}}
            \item \math{x \in \mathbb{Z}, y \in \mathbb{Z}}
            \item \math{x \in \mathbb{Q}, y \in \mathbb{Q}}
            \item \math{x \in \mathbb{R}, y \in \mathbb{R}}
        \end{itemize}
    \item \math{x \in \mathbb{R}}; \math{y \in \mathbb{N}}
\end{enumerate}

\section{Problem 4.9}
\begin{enumerate}[(a)]
    \item set "\math{{n}^{3} + 5} is odd" as \math{p}, "\math{n} is even" as \math{q}
    \begin{itemize}
        \item Direct Prove
            \begin{enumerate}[i)]
             \item set \math{p} as true
             \item \math{{n}^{3} + 5 = 2k + 1}, while \math{k} is integer
             \item \math{{n}^{3}} is even
             \item \math{{n}^{3} = n \times n \times n}, since an even result can only come from even number times an even number, and a number can only either be even or odd, so \math{n} is even
             \item \math{p \to q} is true \math{\hfill{\blacksquare}}
            \end{enumerate}
        \item Contraposition
             \begin{enumerate}[i)]
                 \item set \math{q} as false
                 \item \math{n} is odd
                 \item \math{{n}^3 = n \times n \times n}. Since odd number times an odd number will end in odd result, \math{{n}^{3}} is odd
                 \item an odd number add an odd number will end in even result, so \math{{n}^{3} + 5} is even
                 \item \math{p} is false
                 \item \math{p \to q} is true \math{\hfill{\blacksquare}}
            \end{enumerate}
    \end{itemize}
    \item set "3 does not divide \math{n}" as \math{p}, "3 divides \math{{n}^{2} + 2}" as \math{q}
        \begin{itemize}
            \item Direct Prove
                \begin{enumerate}[i)]
                    \item set \math{p} as true
                    \item \math{
                        \begin{cases}
                            \math{n = 3k + 1} \\
                            \math{n = 3k + 2}
                        \end{cases}
                    }, k is integer
                    \item
                        \math{
                            \begin{cases}
                                9{k}^{2}+6k+1+2 \\
                                9{k}^{2}+6k+4+2
                            \end{cases} \Rightarrow
                            \begin{cases}
                                3(3{k}^{2}+2k+1) \\
                                3(3{k}^{2}+2k+2)
                            \end{cases}
                        }, under both conditions, \math{{n}^2 + 2} is a multiple of 3
                    \item \math{q} is true
                    \item \math{p \to q} is true \math{\hfill{\blacksquare}}
                \end{enumerate}
            \item Contraposition
                \begin{enumerate}[i)]
                    \item set \math{q} as false
                    \item for \math{n}, \math{k} as integer, there are three condition: \math{
                        \begin{cases}
                        n = 3k \\
                        n = 3k + 1\\
                        n = 3k + 2
                        \end{cases}
                    }
                    \item \math{n^2 + 2 \Rightarrow
                        \begin{cases}
                            9{k}^{2} + 2\\
                            9{k}^{2} + 6k + 3  = 3(3{k}^{2} + 2k + 1)\\
                            9{k}^{2} + 6k + 6 = 3(3{k}^{2} + 2k + 2)
                        \end{cases}
                    }
                    \item Only \math{9{k}^{2} + 2} match \math{\neg q}
                    \item \math{n = 3k}, \math{n} divides 3, \math{p} is false
                    \item \math{p \to q} is true 
                    \math{\hfill{\blacksquare}}
                \end{enumerate}
        \end{itemize}
\end{enumerate}


\section{Problem 4.15}
\begin{itemize}
    \item [(e)] Direct Prove; set \math{k \in \mathbb{Z}}, \math{p} as "\math{n^2 + 3n + 4} is even"
        \begin{enumerate}
            \item \math{n = \begin{cases}
                            2k \\
                            2k + 1
                        \end{cases}
                        }
            \item \math{n^2 + 3n + 4 = 
                        \begin{cases}
                            4k^2 + 6k + 4 \\
                            4{k}^2 + 10k + 8 \\ 
                        \end{cases} \Rightarrow 
                        \begin{cases}
                            2(2{k}^2 + 3k + 2) \\
                            2(2{k}^2 + 5k + 4)
                        \end{cases}
                        }
            \item under all conditions, the result of \math{n^2 + 3n + 4} can be divided by 2
            \item \math{p} is true
            \hfill{\math{\blacksquare}}
        \end{enumerate}
    \item [(w)] Contraposition; set \math{p} as "\math{a} and \math{b} are positive real numbers with \math{ab < 10000}, \math{q} as "\math{min(a, b) < 100}"
        \begin{enumerate}
            \item set \math{q} as false
            \item \math{min(a, b) \geq 100}
            \item the minimum value of \math{a, b} is 100
            \item \math{ab = 100 \times 100 = 10000}, \math{10000 \nless 10000}
            \item \math{p} is false
            \item \math{p \to q} must be true 
            \hfill{\math{\blacksquare}}
        \end{enumerate}
\end{itemize}

\section{Problem 4.26}

\begin{center}
    \begin{center}
        \textbf{Truth table}
    \end{center}
    \begin{tabular}{|c|c|c|c|}
        \hline
        p  & q  & \math{p \to q} & \math{\neg (p \to q)} \\
        \hline
        T  & T  & T & F\\
        T  & F  & F & T\\
        F  & T  & T & F\\
        F  & F  & T & F\\
        \hline
    \end{tabular}
\end{center} 

\begin{itemize}
    \item [(b)]  
        \begin{itemize}
            \item \textbf{Prove.} (Direct Prove) From the truth table above, in order to prove \math{\neg (p \to q)} true, we need to prove \math{p} as true and \math{q} as false
            \item \textbf{Disprove.} (show a counterexample) From the truth table above, in order to prove \math{p \to q}, for example, we can make \math{p} as true, and \math{q} as true, so \math{\neg (p \to q)} will always be false
        \end{itemize}
    \item [(d)] \math{\forall x : ((\forall n : P(n)) \to Q(x))}
        \begin{itemize}
            \item \textbf{Prove.} (Show for general object) Showing that \math{P(n)} is false for all \math{n}, then based on the truth table, the claim will always be true.
            \item \textbf{Disprove.} (show a counterexample) Given an pair of \math{n, x} that makes \math{P(n)} as true while \math{Q(x)} is false.
        \end{itemize}
    \item [(f)] \math{\exists x : ((\exists n : P(n)) \to Q(x))}
        \begin{itemize}
            \item \textbf{Prove.} (Show an example) Given an pair of \math{x, n} that makes \math{P(n)} as true while \math{Q(x)} is true.
            \item \textbf{Disprove.} (show an counterexample) Given an \math{n} that makes \math{P(n)} as true while there is no corresponded \math{x} which makes \math{Q(x)} is false.
        \end{itemize}
\end{itemize}

\section{Problem 4.36(j)}

\def\firstcircle{(0,0) circle (1) (0,1)}
\def\secondcircle{(1,0) circle (1) (1,1)}
\def\totalrectangle{(-2,-2) rectangle (3,2)}
\def\rectangletextbox{(0,0) rectangle (0,0)}

Set \math{P_1}, \math{P_2}, \math{P_3} to the corresponded areas shown below. 

\begin{center}
    \begin{tikzpicture}[text=black]
         \draw \firstcircle node [above] {\math{A}}
         \secondcircle node [above] {\math{B}}
         \totalrectangle node [] {}
         \rectangletextbox node [midway, left=0.5em] {\math{P_1}}
         \rectangletextbox node [midway, right=0.5em] {\math{P_2}}
         \rectangletextbox node [midway, right=3em] {\math{P_3}};
    \end{tikzpicture}
\end{center}

\begin{enumerate}[(1)]
    \item 
        \begin{center}
        \centering
            \begin{minipage}{.4\textwidth}
                \centering
                \begin{tikzpicture}[text=black]
                    % Fill
                    \begin{scope}
                        \fill[green] \firstcircle;
                        \clip \totalrectangle
                              \firstcircle;
                    \end{scope}
                     \draw \firstcircle node [above] {\math{A}}
                     \secondcircle node [above] {\math{B}}
                     \totalrectangle node [midway, above=6em] {\math{|A| \Rightarrow P_1 + P_2}};
                \end{tikzpicture}
            \end{minipage} % graph 1
            \begin{minipage}{.4\textwidth}
            \centering
                \begin{tikzpicture}
                    % Fill
                    \begin{scope}
                        \fill[blue] \secondcircle;
                        \clip \totalrectangle
                              \firstcircle;
                    \end{scope}
                     \draw 
                     \firstcircle node [text=black,above] {\math{A}}
                     \secondcircle node [text=black,above] {\math{B}}
                     \totalrectangle node [text=black,midway, above=6em] {\math{|B| \Rightarrow P_2 + P_3}};
                \end{tikzpicture}
            \end{minipage} % graph 2
        \end{center}
    \item
        \begin{center}
        \centering
            \begin{minipage}{.4\textwidth}
                \centering
                \begin{tikzpicture}
                    \begin{scope}[fill opacity=0.5]
                        \fill[green] \firstcircle;
                        \fill[blue] \secondcircle;
                        \clip \totalrectangle
                              \firstcircle;
                    \end{scope}
                \draw 
                    \firstcircle node [text=black,above] {\math{A}}
                    \secondcircle node [text=black,above] {\math{B}}
                    \totalrectangle node [text=black,midway, above=6em] {\math{A \cup B \Rightarrow P_1 + P_2 + P_3}};
                \end{tikzpicture}
            \end{minipage} % graph 3
            \begin{minipage}{.4\textwidth}
            \centering
                \begin{tikzpicture}
                    \begin{scope}[even odd rule]% first circle without the second
                        \clip 
                            \firstcircle
                            \secondcircle
                            \totalrectangle;
                        \fill[green!50!blue] \firstcircle;
                    \end{scope}
                    \draw 
                        \firstcircle node [text=black,above] {\math{A}}
                        \secondcircle node [text=black,above] {\math{B}}
                        \totalrectangle node [text=black,midway, above=6em] {\math{A \cap B \Rightarrow P_2}};
                \end{tikzpicture}
            \end{minipage} % graph 4
        \end{center}
    \item From the graph above, \math{(P_1 + P_2) + (P_2 + P_3) =  (P_1 + P_2 + P_3) + (P_2) = P_1 + 2{P_2} + P_3}. \\\\ The equation is true. \math{\hfill{\blacksquare}}
\end{enumerate}

\section{Problem 4.45(b)}

\begin{enumerate}[i)]
    \item Set \math{a = 1}; set Definition (2) as \math{p}
        \begin{enumerate}[(1)]
            \item a function can either approach \math{\infty} or \math{a}
            \item from (ii), \math{f(n) \to 1}
            \item \math{f(n)} will not approach \math{\infty}
            \item \math{p} is false 
            \math{\hfill{\blacksquare}}
        \end{enumerate}
    % \item set Definition (1) as \math {p}; \math{n = 1, 2}, \math{C = 3}
    %     \begin{enumerate}[(1)]
    %         \item when \math{n = 1}, \math{f(n) = 2}
    %         \item when \math{n = 2}, \math{f(n) < 2}
    %         \item The function is decreasing over time
    %         \item Since \math{2 < C}, \math{f(n) < C}
    %         \item \math{p} is false
    %     \end{enumerate}
    \item Set \math{a = 1}; set Definition (2) as \math{p}
        \begin{enumerate}[(1)]
            \item Assume \math{\neg p} (contradiction)
            \item \math{f(n) \to a} if for any \math{\epsilon > 0}, there is \math{n_{\epsilon}} such that for all \math{n \geq n_{\epsilon}}, \math{|f(n) - a| > \epsilon}.
            \item \math{\frac{n+3}{n+1} > \epsilon \Rightarrow \frac{2}{\epsilon} - 1 > n}
            \item set \math{\epsilon = 2}, \math{2/2 - 1 = 0}
            \item \math{0 \leq 1}, which is a \textbf{fishy} conclusion
            \item \math{p} has to be true when \math{a = 1}
            \math{\hfill{\blacksquare}}
        \end{enumerate}
    \item Set \math{a = 2}, \math{\epsilon = 0.1}, \math{n = 2}; set Definition (2) as \math{p}
        \begin{enumerate}[(1)]
            \item when \math{n = 2}, \math{f(n) = \frac{5}{3}}
            \item \math{|f(n) - 2| = \frac{1}{3}}
            \item \math{|f(n) - a| > \epsilon} 
            \item \math{p} is false when \math{a = 2} 
            \math{\hfill{\blacksquare}}
        \end{enumerate} 
\end{enumerate}

\section{Problem 5.7(f)}
Define the claim P(n): \math{(1 - \frac{1}{2})(1 - \frac{1}{3})...(1 - \frac{1}{n}) = \frac{1}{n}}

\begin{enumerate}[i)]
    \item \textbf{Base case} \math{P(1) = 1 - \frac{1}{2} = \frac{1}{2}}
    \item \textbf{Induction Step} Showing \math{P(n) \to P(n+1)} for all \math{n \geq 2}, using a direct proof
    \begin{enumerate}[(1)]
        \item Assume \math{P(n)} is \math{T}: \math{(1 - \frac{1}{2})(1 - \frac{1}{3})...(1 - \frac{1}{n}) = \frac{1}{n}}
        \item Show \math{P(n+1)} is \math{T}: \math{(1 - \frac{1}{2})(1 - \frac{1}{3})...(1 - \frac{1}{n})(1 - \frac{1}{n+1}) = \frac{1}{n+1}}
        \item 
            \begin{equation}
                \begin{split}
                    (1 - \frac{1}{2})(1 - \frac{1}{3})...(1 - \frac{1}{n})(1 - \frac{1}{n+1}) & = [(1 - \frac{1}{2})(1 - \frac{1}{3})...(1 - \frac{1}{n})](1 - \frac{1}{n+1}) \\
                    & = (\frac{1}{n})(1 - \frac{1}{n+1}) \\
                    & = \frac{n + 1 - 1}{n(n+1)} \\
                    &= \frac{1}{n+1}
                \end{split}
            \end{equation}
        \item This is exactly what was to be shown. So, \math{P(n+1)} is \math{T}
    \end{enumerate}
     \hfill{\math{\blacksquare}}
\end{enumerate}
\end{document}
